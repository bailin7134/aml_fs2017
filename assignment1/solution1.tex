% --------------------------------------------------------------
% This is all preamble stuff that you don't have to worry about.
% Head down to where it says "Start here"
% --------------------------------------------------------------

\documentclass[12pt]{article}

\usepackage[margin=1in]{geometry} 
\usepackage{amsmath,amsthm,amssymb}
\usepackage{color}
\usepackage{tikz, pgfplots}
\usepackage{graphicx}
\usepackage{epstopdf} %converting to PDF
\usepackage{subcaption}
\usepackage{listings}

\makeatletter

\renewcommand\section{\@startsection {section}{1}{\z@}%
	{-3.5ex \@plus -1ex \@minus -.2ex}%
	{2.3ex \@plus.2ex}%
	{\normalfont\large\bfseries}}% from \Large
\renewcommand\subsection{\@startsection{subsection}{2}{\z@}%
	{-3.25ex\@plus -1ex \@minus -.2ex}%
	{1.5ex \@plus .2ex}%
	{\normalfont\large\bfseries}}% from \large
\makeatother

\begin{document}
	
	% --------------------------------------------------------------
	%                         Start here
	% --------------------------------------------------------------
	
	%\renewcommand{\qedsymbol}{\filledbox}
	
	\title{\textbf{Advanced Topics in Machine Learning Exercise \#{1}}\\
	Universit{\"a}t Bern}%replace X with the appropriate number
	\author{{Lin Bai, 09935404}} %replace with your name
	
	\maketitle

	%%%%%%%%%%%%%%%%%%%%%%%%%%%%%%%%%%%%%%%%%%%%%%%%%%%%%%%%%%%%%%%%%%%%%%%%%%%%%%%%%%%
	%%%%%%   question 1
	%%%%%%%%%%%%%%%%%%%%%%%%%%%%%%%%%%%%%%%%%%%%%%%%%%%%%%%%%%%%%%%%%%%%%%%%%%%%%%%%%%%
	\section{Solution to question 1}
	The object function and stochastic gradient descent update rule are\\
	\\
	$\ell(\theta) = \log L(\theta) = \sum_{i=1}^{m}y{(i)}\log h(x^{(i)})+(1-y^{(i)})\log (1-h(x^{(i)}))$\\
	\\
	\noindent
	$\theta_j := \theta_j + \alpha(y^{(i)}-h_{\theta}(x^{(i)}))x_j^{(i)}$\\
	\noindent
	where
	$h_{\theta} = \frac{1}{1+\exp(\theta^T x)}$
	
	%%%%%%%%%%%%%%%%%%%%%%%%%%%%%%%%%%%%%%%%%%%%%%%%%%%%%%%%%%%%%%%%%%%%%%%%%%%%%%%%%%%
	%%%%%%   question 2
	%%%%%%%%%%%%%%%%%%%%%%%%%%%%%%%%%%%%%%%%%%%%%%%%%%%%%%%%%%%%%%%%%%%%%%%%%%%%%%%%%%%
	\section{Solution to question 2}
	The code of stochastic gradient descent for logistic regression is listed below.\\
	\lstset{language=R}
	\lstset{frame=lines}
	\lstset{caption={code of logistic regression using stochastic gradient descent}}
	\lstset{label={lst:code_direct}}
	\lstset{basicstyle=\footnotesize\ttfamily}
	\begin{lstlisting}[breaklines=true]
	function sigmoid(z)
		-- 1/1+exp(-z)
		g = torch.FloatTensor(1,1):fill(1):cdiv(1.0+torch.exp(-z))
		return g
	end
	
	function train_logistic_sgd(data, labels)
		-- learning rate
		local alpha = torch.FloatTensor(1,1):fill(0.00005)
	
		local p = data:size(1)  -- size of the features
		local n = data:size(2)  -- number of train samples
		-- initialize the value of theta
		local theta = torch.FloatTensor(1,p):fill(0)
	
		for k = 1, n, 1 do
			i = math.random(1,n)
			gradJ = data[{{}, {i}}] * (labels[i] - sigmoid(theta * data[{{}, {i}}])):view(1, 1)
			theta = theta + gradJ * alpha:view(1, 1)
		end
		return theta
	end
	\end{lstlisting}
	\noindent
	When learning rate $\alpha = 0.00005$, the average precision 92.45 is achieved.\\
	\\
	\noindent
	The precision-recall plot is shown below\\
	\begin{figure}[htpb]
		\centering
		\includegraphics[width=0.5\textwidth]{precision_recall.png}
		\caption{precision recall plot}
	\end{figure}
	%%%%%%%%%%%%%%%%%%%%%%%%%%%%%%%%%%%%%%%%%%%%%%%%%%%%%%%%%%%%%%%%%%%%%%%%%%%%%%%%%%%
	%%%%%%   question 3
	%%%%%%%%%%%%%%%%%%%%%%%%%%%%%%%%%%%%%%%%%%%%%%%%%%%%%%%%%%%%%%%%%%%%%%%%%%%%%%%%%%%
	\section{Solution to question 3}
	\lstset{caption={roc curve plot}}
	\begin{lstlisting}[breaklines=true]
	tpr = recall
	fpr = torch.cdiv(recall, precision)-recall
	gnuplot.pngfigure('ROC.png')
	gnuplot.plot(fpr, tpr)
	gnuplot.xlabel('false positive rate')
	gnuplot.ylabel('true positive rate')
	\end{lstlisting}
	\begin{figure}[htpb]
		\centering
		\includegraphics[width=0.5\textwidth]{ROC.png}
		\caption{ROC plot}
	\end{figure}
	%%%%%%%%%%%%%%%%%%%%%%%%%%%%%%%%%%%%%%%%%%%%%%%%%%%%%%%%%%%%%%%%%%%%%%%%%%%%%%%%%%%
	%%%%%%   question 4
	%%%%%%%%%%%%%%%%%%%%%%%%%%%%%%%%%%%%%%%%%%%%%%%%%%%%%%%%%%%%%%%%%%%%%%%%%%%%%%%%%%%

	

	
	% --------------------------------------------------------------
	%     You don't have to mess with anything below this line.
	% --------------------------------------------------------------
	
\end{document}